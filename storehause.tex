\begin{proposition}
	For each polynomial $A$,
	there are natural isomorphisms
	$y\trleft A\cong A$
	and
	$A\trleft y\cong A$ in $\Poly$.
\end{proposition}
\begin{proposition}
	\label{prop:EmbInducesMoncat}
	Let $\one{V}$ be a category and $\one{W}=(\one{W},\otimes',I')$ be a monoidal category.
	Consider a functor $\otimes\colon\one{V}\times\one{V}\arr\one{V}$ and an object $I\in\one{V}$.
	Suppose that we are given a functor $F\colon\one{V}\arr\one{W}$
	such that there are a family of isomorphisms $F(A\otimes B)\cong F(A)\otimes F(B)$ natural in $A,B\in\one{V}$
	and an isomorphism $F(I)\cong I$.
	If $F$ induces a fully faithful functor on the core groupoids,
	then $(\one{V},\otimes,I)$ becomes a monoidal category and $F$ is a pseudo monoidal functor.
\end{proposition}
\begin{definition}
	Define
	$\Set\arr[hook]\Poly$
	as a functor induced
	from the 2-natural transformation
	$\Delta(\one{1})\arr[Rightarrow]\left[-,\Set\right]^\op$
	whose component at $S\in\Set^\op$ is the name for the terminal object in $\left[S,\Set\right]^\op$.
	In other words, the fibration $\Poly\arr"{(-)_\ob}"\Set$ has the fibred terminal.
	This functor is automatically fully faithful, and we regard sets as polynomials through this functor.
\end{definition}
\begin{lemma}
	\label{lem:PolyMonoidalCaty}
	There is a family of isomorphisms
	$\rep(P\trleft Q)\cong\rep(P)\circ\rep(Q)$ natural in $P,Q\in\Poly$.
	Moreover, we have $\rep(y)\cong\id_\Set$ in $\left[\Set,\Set\right]$.
\end{lemma}
\begin{proposition}
	\label{prop:repIsEmb}
	$\rep$ induces a fully faithful functor on the core groupoids.
\end{proposition}
\begin{proof}
	Suppose we are given
	$P,Q\in\Poly$,
	an isomorphism $\alpha\colon\rep(P)\cong\rep(Q)$ in $\left[\Set,\Set\right]$,
	and
	$S\in\Set$.
	We have the following commutative diagram:
	\[
		\begin{tikzcd}
			\sum_{p\in P_\ob}[P_p,S]
			\ar[r,"\cong"description,phantom]
			\ar[d,""]
				&
				P\trleft S
				\ar[r,"\alpha_S",weq']
				\ar[d,"P\trleft!_S"']
					&
					Q\trleft S
					\ar[d,"Q\trleft!_S"]
					\ar[r,"\cong"description,phantom]
						&
						\sum_{q\in Q_\ob}[Q_q,S]
						\ar[d,""]
			\\
			P_\ob
			\ar[r,"\cong"description,phantom]
				&
				P\trleft 1
				\ar[r,"\alpha_1"',weq]
					&
					Q\trleft 1
					\ar[r,"\cong"description,phantom]
						&
						Q_\ob
		\end{tikzcd}
	\]
	where the left and the right vertical arrows are projections. Therefore, we have $\left[P_p,S\right]\cong\left[Q_{\alpha_1(p)},S\right]$
	for each $p\in P_\ob$ and $S\in\Set$.
	By applying the yoneda lemma, we obtain $Q_{\alpha_1(p)}\cong P_p$ for each $p\in P_\ob$, which induces an isomophism $\bar\alpha\colon P\cong Q$ in $\Poly$.
	One can readily check that $\bar\alpha$ is a unique isomorphism satisfying $\rep(\bar\alpha)=\alpha$.
\end{proof}

\begin{definition}
	Define a functor $\rep\colon\Poly\arr\left[\Set,\Set\right]$
	as the currying of the functor $\Poly\times\Set\arr"\trleft"\Set$ obtained in \Cref{lem:trleftRestSets}.
\end{definition}
