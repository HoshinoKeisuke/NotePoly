\documentclass[a4paper,dvipsnames, 11pt]{amsart} %pdf latex
\usepackage{preamble}
\usepackage{lipsum}

\begin{document}
\maketitle
\begin{notation}
	We employ the following notations.
	\begin{itemize}
		\item %
			Sets are regarded as discrete categories,
			and categories are regarded as locally discrete 2-categories.
		\item %
			$\Set$ is the large category of small sets.
		\item %
			We write $\CATbi$ for the huge 2-category of large categories.
		\item %
			For each set $S$ and a functor $X\colon S\arr\Set$,
			we write $X_s$ for the image of $s\in S$ under $X$.
			Moreover, we write $(X_s)_{s\tcolon S}$ for $X$.
		\item %
			For each set $S$ and functors $X,Y\colon S\arr \Set$,
			a natural transformation $f\colon X \arr[Rightarrow]Y$ is denoted as a family of functions
			$(f_s)_{s\tcolon S}$.
		\item %
			We write $1$ for the terminal set, and we write $*$ for the unique element in $1$.
	\end{itemize}
\end{notation}
\section{polynomials}
\begin{definition}
	We define a category $\Poly$ by the Grothendieck construction of the following 2-functor.
	\[
		\Set^\op
		\arr"\left[-,\Set\right]^\op"[][5]
		\CATbi
	\]
	A \emph{polynomial} is an object in $\Poly$.
	We write
	\[
		(-)_\ob\colon\Poly\arr\Set
	\]
	for the fibration corresponding to the 2-functor.
	For each polynomial $A$,
	we write $A=\sum_{a\tcolon A_\ob}y^{A_a}$.

	In particular, we mean by $y$ the polynomial satisfying
	$(y)_\ob=1$ and $(y)_*=1$.
\end{definition}
\begin{definition}
	We define a category $\Fam$ by the Grothendieck construction of the following 2-functor.
	\[
		\Set^\op
		\arr"\left[-,\Set\right]"[][5]
		\CATbi
	\]
	Note that the fibration $\Poly\arr\Set$ is the fibrewise opposit of $\Fam\arr\Set$.
\end{definition}
\begin{definition}
	Consider the composite
	\[
		\begin{tikzcd}[row sep=5mm]
			\left[S,\Set\right]^\op\times\Set
			\ar[rr,"{\langle\id_{[S,\Set]},\Delta\circ\proj\rangle}"]
				&\ &
				\left[S,\Set\right]^\op\times\left[S,\Set\right]
				\ar[r]
					&
					\left[S,\Set\right]
			\\
			((A_s)_{s\tcolon S}, T)
			\ar[rrr,mapsto]
			\ar[u,"\in"marking,phantom]
				&\ &
					&
					(\left[A_s,T\right])_{s\tcolon S}
					\ar[u,"\in"marking,phantom]
		\end{tikzcd}
	\]
	where $\Delta\colon\Set\arr\left[S,\Set\right]$ in the former is the diagonal and
	the latter is the internal hom for $[S,Set]$.
	This is natural in $S\in\Set^\op$ and induces a functor
	$\Poly\times\Set\arr\Fam$.
	We write $\rep\colon\Poly\arr\left[\Set,\Set\right]$ for the currying of the composite
	\[
		\begin{tikzcd}[row sep=5mm]
			\Poly\times\Set
			\ar[r]
				&
				\Fam
				\ar[r,"\Sigma"]
					&
					\Set
			\\
			(A,S)
			\ar[rr,mapsto]
			\ar[u,"\in"marking,phantom]
				&
					&
					\sum_{a\tcolon A_\ob}\left[A_p,S\right]
					\ar[u,"\in"marking,phantom]
		\end{tikzcd}
	\]
	where $\sum$ is the coproduct (i.e., the left adjoint of the fibred terminal $\Set\arr\Fam$).
\end{definition}
\begin{proposition}
	\label{prop:repIsFF}
	$\rep$ is fully faithful.
\end{proposition}
\begin{proof}
	Suppose we are given
	$P,Q\in\Poly$,
	a morphism $\alpha\colon\rep(P)\arr\rep(Q)$ in $\left[\Set,\Set\right]$,
	and
	$S\in\Set$.
	We have the following commutative square:
	\[
		\begin{tikzcd}
			\sum_{p\in P_\ob}[P_p,S]
			\ar[r,"\alpha_S"]
			\ar[d,"\rep(P)(!_S)"']
				&
				\sum_{q\in Q_\ob}[Q_q,S]
				\ar[d,"\rep(Q)(!_S)"]
			\\
			P_\ob
			\ar[r,"\alpha_1"']
				&
				Q_\ob
		\end{tikzcd}
	\]
	where the left and the right vertical arrows are projections. Therefore, we have a family of morphisms
	$\left[P_p,S\right]\arr\left[Q_{\alpha_1(p)},S\right]$
	natural in $p\in P_\ob$ and $S\in\Set$.
	By applying the yoneda lemma, we obtain $Q_{\alpha_1(p)}\arr P_p$ for each $p\in P_\ob$, which induces a morphism $\bar\alpha\colon P\arr Q$ in $\Poly$.
	One can readily check that $\bar\alpha$ is a unique morphism satisfying $\rep(\bar\alpha)=\alpha$.
\end{proof}
\begin{lemma}
	\label{lem:repClosedunderMonStr}
	(The essential image of) $\rep$ is closed under the monoidal structure $(\circ,\id_\Set)$ on $[\Set,\Set]$.
\end{lemma}
\begin{proof}
	$\rep(y)\cong\id_\Set$ is trivial. It suffices to show $\rep$ is closed under $\circ$.
	Let $A,B$ be polynomials. Let us define another polynomial $A\trleft B$ as follows.
	\[
		(A\trleft B)_\ob
		=\sum_{a\tcolon A_\ob}\left[A_a,B_\ob\right]
	\]
	\[
		(A\trleft B)_{a,t}
		=\sum_{u\tcolon A_a} B_{t(u)}
	\]
	for any
	$a\tcolon A_\ob$ and $t\colon A_a\arr B_\ob$.
	Then, observe the following bijections.
	\begin{align*}
		\rep(A\trleft B)(S)
		&
		\cong
		\sum_{a\tcolon A_\ob}\sum_{t\tcolon\left[A_a,B_\ob\right]}\left[\sum_{u\tcolon A_a}B_{t(u)},S\right]
		\\
		&
		\cong
		\sum_{a\tcolon A_\ob}\sum_{t\tcolon\left[A_a,B_\ob\right]}\prod_{u\tcolon A_a}\left[B_{t(u)},S\right]
		\\
		&
		\cong
		\sum_{a\tcolon A_\ob}\prod_{u\tcolon A_a}\sum_{b\tcolon B_\ob}\left[B_{b},S\right]
		\\
		&
		\cong
		\sum_{a\tcolon A_\ob}\left[A_a,\sum_{b\tcolon B_\ob}\left[B_{b},S\right]\right]
		\\
		&
		=
		\rep(A)(\rep(B)(S))
	\end{align*}
	The third bijection is the distributive law.
	Since these are natural in $S\in\Set$ this shows that $\rep(A\trleft B)\cong\rep(A)\circ\rep(B)$.
\end{proof}
\begin{proposition}
	\label{prop:FFInducesMoncat}
	Let $\one{V}$ be a category and $\one{W}$ be a monoidal category.
	Suppose there is a fully faithful functor $F\colon\one{V}\arr[hook]\one{W}$
	that is closed under the monoidal structure on $\one{W}$.
	Then there exists a monoidal structure on $\one{V}$ that makes $F$ strong monoidal,
	which is unique up to isomorphisms.
\end{proposition}
\begin{proof}
	\comment{folklore}
\end{proof}
\begin{proposition}
	\label{prop:PolyMonoidalCaty}
	$(\Poly,\trleft,y)$ is a monoidal category
	such that $\rep\colon\Poly\arr\left(\left[\Set,\Set\right],\circ,\id_\Set\right)$ is a pseudo monoidal functor.
\end{proposition}
\begin{proof}
	Follows from \Cref{lem:repClosedunderMonStr,prop:repIsFF,prop:FFInducesMoncat}.
\end{proof}
\begin{remark}
	The functor $-\trleft-\colon\Poly\times\Poly\arr\Poly$ acts on morphisms as follows.
	Let $F\colon A\arr X$, $G\colon B\arr Y$ be morphisms in $\Poly$.
	\[
		(F\trleft G)_\ob
		\colon
		\sum_{a\tcolon A_\ob}\left[A_a,B_\ob\right]
		\arr
		\sum_{x\tcolon X_\ob}\left[X_x,Y_\ob\right]
		\colon
		(a,t)\mapsto\left(F_\ob(a),F_{a}\fatsemi t\fatsemi G_{\ob}\right)
	\]
	\[
		(F\trleft G)_{a,t}\colon
		\sum_{v\tcolon X_{F_\ob(a)}}Y_{G_\ob(t(F_{a}(v)))}
		\arr
		\sum_{u\tcolon A_{a}}B_{t(u)}
		\colon
		(v,r)
		\mapsto
		(F_a(v),G_{t(F_a(v))}(r))
	\]
\end{remark}
\begin{lemma}
	\label{lem:trleftRestSets}
	$\Poly\times\Poly\arr"\trleft"\Poly$
	restricts to a functor
	$\Poly\times\Set\arr"\trleft"\Set$
	and coincides with the uncurrying of $\rep$.
\end{lemma}
\begin{proof}
\end{proof}

\section{Double categories}
By \emph{double categories}, we mean psuedo ones.
\begin{definition}
	A double category $\dbl{D}$ is a \emph{company} if
	its tight arrows have companions.
	$\dbl{D}$ is an \emph{opcompany} if $\dbl{D}^\hop$ is a company.
	We say $\dbl{D}$ is an \emph{equipment} if $\dbl{D}$ and $\dbl{D}^\hop$ are companies.
\end{definition}
\begin{definition}
	\comment{(It is defined for arbitrary bvdc whose underlying sdc is a opcompany.)}
	Let $\dbl{D}$ be an opcompany.
	Define another double category $\Ret(\dbl{D})$ as follows.
	\begin{itemize}
		\item %
			The vertical category is the same as $\dbl{D}$.
		\item %
			A loose arrow is a loose arrow in $\dbl{D}$.
		\item %
			A cell
			\[
				\begin{tikzcd}
					A
					\sar[r,"p"]
					\ar[d,"f"']
						&
						B
						\ar[d,"g"]
					\\
					X
					\sar[r,"q"']
						&
						Y
				\cellsymb"\alpha"{1-1}{2-2}
				\end{tikzcd}
				\incat{\Ret(\dbl{D})}
			\]
			is a cell in $\dbl{D}$ of the following form.
			\[
				\begin{tikzcd}
					X
					\sar[r,"q"]
					\ar[d,equal]
						&
						Y
						\sar[r,"g^*"]
							&
							B
							\ar[d,equal]
					\\
					X
					\sar[r,"{f^*}"']
						&
						A
						\sar[r,"p"']
							&
							B
				\cellsymb"\alpha"{1-1}{2-3}
				\end{tikzcd}
			\]
		\item %
			\comment{horizontal composite}
		\item %
			\comment{vertical composite}
		\qedhere %
	\end{itemize}
\end{definition}
\begin{proposition}
	\label{prop:replVDC}
	The virtual double category of monoids in $\Ret(\dbl{D})^\vop$ is representable if $\dbl{D}$ has local reflexive coequalizers.
\end{proposition}
\begin{definition}
	Let $\dbl{D}$ be a double category with local reflexive coequalizers.
	\begin{itemize}
		\item %
			We write $\Mod(\dbl{D})$ for the double category corresponding to the representable virtual double category of monoids in $\Ret(\dbl{D})$.
		\item %
			We write $\Mod_{\co}(\dbl{D})$ for the vertical opposite of the double category corresponding to the representable virtual double category of monoids in $\Ret(\dbl{D})^\vop$
			obtained in \Cref{prop:replVDC}.
		\qedhere %
	\end{itemize}
\end{definition}
\begin{lemma}
	Let $\dbl{D}$ be a double category with local reflexive coequalizers.
	There is an isomorphism $\bi{H}(\Mod_\co(\dbl{D}))\cong\bi{H}(\Mod(\dbl{D}))$.
	\comment{Where?}
\end{lemma}
\begin{definition}
\end{definition}

\bibliographystyle{halpha-abbrv}
\bibliography{bibliography}
\end{document}
