\documentclass[a4paper,dvipsnames, 11pt]{amsart}
\usepackage{preamble}
\usepackage{lipsum}

\begin{document}
\maketitle
\begin{notation}
	We employ the following notations.
	\begin{itemize}
		\item %
			Sets are regarded as discrete categories,
			and categories are regarded as locally discrete 2-categories.
		\item %
			$\Set$ is the large category of small sets.
		\item %
			We write $\CATbi$ for the huge 2-category of large categories.
		\item %
			For each set $S$ and a functor $X\colon S\arr\Set$,
			we write $X_s$ for the image of $s\in S$ under $X$.
			Moreover, we write $(X_s)_{s\tcolon S}$ for $X$.
		\item %
			For each set $S$ and functors $X,Y\colon S\arr \Set$,
			a natural transformation $f\colon X \arr[Rightarrow]Y$ is denoted as a family of functions
			$(f_s)_{s\tcolon S}$.
		\item %
			We write $1$ for the terminal set, and we write $*$ for the unique element in $1$.
	\end{itemize}
\end{notation}
\section{polynomials}
\begin{definition}
	We define a category $\Poly$ as the Grothendieck construction of the following 2-functor.
	\[
		\Set^\op
		\arr"\left[-,\Set\right]^\op"[][5]
		\CATbi
	\]
	A \emph{polynomial} is an object in $\Poly$.
	We write
	\[
		(-)_\ob\colon\Poly\arr\Set
	\]
	for the fibration corresponding to the 2-functor.
	For each polynomial $A$,
	we write $A=\sum_{a\tcolon A_\ob}y^{A_a}$.

	In particular, we mean by $y$ the polynomial satisfying
	$(y)_0=1$ and $(y)_*=1$.
\end{definition}
\begin{definition}
	Define a functor $-\trleft-\colon\Poly\times\Poly\arr\Poly$ as follows.
	\begin{itemize}
		\item %
			Let $A,B$ be polynomials.
			\[
				(A\trleft B)_\ob
				=\sum_{a\tcolon A_\ob}\left[A_a,B_\ob\right]
			\]
			\[
				(A\trleft B)_{a,t}
				=\sum_{u\tcolon A_a} B_{t(u)}
			\]
		for any
		$a\tcolon A_\ob$ and $t\colon A_a\arr B_\ob$.
		\item %
			Let $F\colon A\arr X$, $G\colon B\arr Y$ be morphisms in $\Poly$.
			\[
				(F\trleft G)_\ob
				\colon
				\sum_{a\tcolon A_\ob}\left[A_a,B_\ob\right]
				\arr
				\sum_{x\tcolon X_\ob}\left[X_x,Y_\ob\right]
				\colon
				(a,t)\mapsto\left(F_\ob(a),F_{a}\fatsemi t\fatsemi G_{\ob}\right)
			\]
			\[
				(F\trleft G)_{a,t}\colon
				\sum_{v\tcolon X_{F_\ob(a)}}Y_{G_\ob(t(F_{a}(v)))}
				\arr
				\sum_{u\tcolon A_{a}}B_{t(u)}
				\colon
				(v,r)
				\mapsto
				(F_a(v),G_{t(F_a(v))}(r))
			\]
		\qedhere %
	\end{itemize}
\end{definition}
\begin{proposition}
	For each polynomial $A$,
	there are natural isomorphisms
	$y\trleft A\cong A$
	and
	$A\trleft y\cong A$ in $\Poly$.
\end{proposition}
\begin{proposition}
	\label{prop:EmbInducesMoncat}
	Let $\one{V}$ be a category and $\one{W}=(\one{W},\otimes',I')$ be a monoidal category.
	Consider a functor $\otimes\colon\one{V}\times\one{V}\arr\one{V}$ and an object $I\in\one{V}$.
	Suppose that we are given a functor $F\colon\one{V}\arr\one{W}$
	such that there are a family of isomorphisms $F(A\otimes B)\cong F(A)\otimes F(B)$ natural in $A,B\in\one{V}$
	and an isomorphism $F(I)\cong I$.
	If $F$ induces a fully faithful functor on the core groupoids,
	then $(\one{V},\otimes,I)$ becomes a monoidal category and $F$ is a pseudo monoidal functor.
\end{proposition}
\begin{definition}
	Define
	$\Set\arr[hook]\Poly$
	as a functor induced
	from the 2-natural transformation
	$\Delta(\one{1})\arr[Rightarrow]\left[-,\Set\right]^\op$
	whose component at $S\in\Set^\op$ is the name for the terminal object in $\left[S,\Set\right]^\op$.
	In other words, the fibration $\Poly\arr"{(-)_\ob}"\Set$ has the fibred terminal.
	This functor is automatically fully faithful, and we regard sets as polynomials through this functor.
\end{definition}
\begin{lemma}
	\label{lem:trleftRestSets}
	$\Poly\times\Poly\arr"\trleft"\Poly$
	restricts to a functor
	$\Poly\times\Set\arr"\trleft"\Set$.
\end{lemma}
\begin{proof}
\end{proof}
\begin{definition}
	Define a functor $\rep\colon\Poly\arr\left[\Set,\Set\right]$
	as the currying of the functor $\Poly\times\Set\arr"\trleft"\Set$ obtained in \Cref{lem:trleftRestSets}.
\end{definition}
\begin{proposition}
	\label{prop:repIsEmb}
	$\rep$ induces a fully faithful functor on the core groupoids.
\end{proposition}
\begin{proof}
	Suppose we are given
	$P,Q\in\Poly$,
	an isomorphism $\alpha\colon\rep(P)\cong\rep(Q)$ in $\left[\Set,\Set\right]$,
	and
	$S\in\Set$.
	We have the following commutative diagram:
	\[
		\begin{tikzcd}
			\sum_{p\in P_\ob}[P_p,S]
			\ar[r,"\cong"description,phantom]
			\ar[d,""]
				&
				P\trleft S
				\ar[r,"\alpha_S",weq']
				\ar[d,"P\trleft!_S"']
					&
					Q\trleft S
					\ar[d,"Q\trleft!_S"]
					\ar[r,"\cong"description,phantom]
						&
						\sum_{q\in Q_\ob}[Q_q,S]
						\ar[d,""]
			\\
			P_\ob
			\ar[r,"\cong"description,phantom]
				&
				P\trleft 1
				\ar[r,"\alpha_1"',weq]
					&
					Q\trleft 1
					\ar[r,"\cong"description,phantom]
						&
						Q_\ob
		\end{tikzcd}
	\]
	where the left and the right vertical arrows are projections. Therefore, we have $\left[P_p,S\right]\cong\left[Q_{\alpha_1(p)},S\right]$
	for each $p\in P_\ob$ and $S\in\Set$.
	By applying the yoneda lemma, we obtain $Q_{\alpha_1(p)}\cong P_p$ for each $p\in P_\ob$, which induces an isomophism $\bar\alpha\colon P\cong Q$ in $\Poly$.
	One can readily check that $\bar\alpha$ is a unique isomorphism satisfying $\rep(\bar\alpha)=\alpha$.
\end{proof}
\begin{lemma}
	\label{lem:PolyMonoidalCaty}
	There is a family of isomorphisms
	$\rep(P\trleft Q)\cong\rep(P)\circ\rep(Q)$ natural in $P,Q\in\Poly$.
	Moreover, we have $\rep(y)\cong\id_\Set$ in $\left[\Set,\Set\right]$.
\end{lemma}
\begin{proposition}
	\label{prop:PolyMonoidalCaty}
	$(\Poly,\trleft,y)$ is a monoidal category
	such that $\rep\colon\Poly\arr\left(\left[\Set,\Set\right],\circ,\id_\Set\right)$ is a pseudo monoidal functor.
\end{proposition}
\begin{proof}
	Follows from \Cref{lem:PolyMonoidalCaty,prop:repIsEmb,prop:EmbInducesMoncat}.
\end{proof}

\bibliographystyle{halpha-abbrv}
\bibliography{bibliography}
\end{document}
